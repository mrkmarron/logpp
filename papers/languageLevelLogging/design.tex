% !TeX root = LanguageLevelLogging.tex
This section describes opportunities, using language, runtime, or compiler support, to address 
general challenges surrounding logging outlined in \autoref{sec:intro}. We can roughly divide 
these into two classes -- performance oriented and functionality oriented. 

\subsection{Logging Performance}
\label{subsec:performancedesign}

\begin{design}
The cost of a \textbf{disabled logging} statement, one that is at a logging level that is disabled, 
should have \textbf{zero-cost} at runtime. This includes both the direct cost of the logging action 
and the indirect cost of of building a format string and processing any arguments. 
\end{design}
For logging frameworks that are included as libraries the compiler/JIT does not, in general, have 
any deep understanding of the enabled/disabled semantics of the logger. As a result it will not be 
able to fully eliminate dead-code associated with disabled logging statements and will pay an, 
individually small but widespread, parasitic cost for these disabled logging statements. These 
costs can be very difficult to diagnose, as they are widely dispersed and individually small, but can 
add up to several percentage points of application runtime.

To avoid these parasitic costs we propose including logging as a set of intrinsics included in the 
specification of the programming language or, if that is not possible, adding compiler/JIT specializations. 
Languages with macros provide a simple limited form of this support by allowing for a compile time flag 
to completely eliminate disabled logging code. However, this solution is limited to code that is explicitly 
contained in the macro and the enabled/disabled levels cannot be changed at runtime. 

Finally, we observe as an additional advantage of lifting log semantics to the language specification level 
allows us to statically verify logging uses. Common errors include format specifier violations~\cite{tyepcheckprintf} 
and accidental state modification in the logging message computation. If the language semantics specify 
logging API's then both of these error classes can be statically checked to avoid runtime errors or hisenbugs 
that appear/disappear when logging levels are changed.

\begin{design}
The cost of an enabled logging statement has two components -- (1) the cost to compute the set of 
arguments to the log statement and (2) the cost to write this data into the log. The cost of computing 
the argument values is, in general unavoidable, and must be done on the hot path of execution. However, 
the cost of (2) should be reduced and/or moved off hot path as much as possible.
\end{design}
Async-logging and immutability based tools asdf...





The major issues this proposal is intended to address are:

\noindent
Lack of high performance logging primitives and fundamental logging challenges.
\begin{enumerate}
 \item Cost of writing data to the log -- particularly with data formatted via 
    `util.inspect` and info such as timestamps.
 \item Ongoing tension between log detail when triaging issues and cost of logging 
    large amounts of 'uninteresting' data.
\item Parasitic costs of disabled logging statements which still execute code to 
    generate dead logging data (e.g., constructing unused strings).
\item Compiler optimizations of both enabled and disabled logging statements.
\end{enumerate}

\subsection{Logging Functionality}
\label{subsec:functionalitydesign}

\noindent
Challenges integrating log data from different sources and difficulty in post processing.
\begin{enumerate}
\item Difficulty in specifying uniform and appropriate logging levels across 
    multiple packages -- and quite possibly multiple logging frameworks.
\item Difficulty in ensuring all logging data is written to a consistent location 
    across multiple packages -- and quite possibly multiple logging frameworks.
\item Correlating fundamental information such as  
    transaction ids, such as async context and HTTP requests, and to log relevant 
    events in the core libraries with user logging data -- same for performance info.
\end{enumerate}

