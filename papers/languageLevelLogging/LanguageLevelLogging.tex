\documentclass[preprint]{sig-alternate-05-2015}

\usepackage{times}
\usepackage{amsmath}
\usepackage{amsfonts}
\usepackage{amssymb}
\usepackage{xspace}
\usepackage{array}
\usepackage{multirow}
\usepackage{balance}
\usepackage{paralist}
\usepackage{graphicx,color}

\usepackage{subfig}

\usepackage{listings} 
\usepackage{color} %use color

\newtheorem{design}{Design Principle}

\definecolor{darkgray}{rgb}{.4,.4,.4}
\definecolor{purple}{rgb}{0.65, 0.12, 0.82}

%define Javascript language
\lstdefinelanguage{JavaScript}{
keywords={typeof, new, true, false, catch, function, return, null, undefined, try, catch, switch, var, if, in, while, do, else, case, break},
keywordstyle=\color{blue}\bfseries,
ndkeywords={class, export, boolean, throw, implements, import, this},
ndkeywordstyle=\color{darkgray}\bfseries,
identifierstyle=\color{black},
sensitive=false,
comment=[l]{//},
morecomment=[s]{/*}{*/},
commentstyle=\color{purple}\ttfamily,
stringstyle=\color{red}\ttfamily,
morestring=[b]',
morestring=[b]"
}
 
\lstset{
language=JavaScript,
extendedchars=true,
basicstyle=\footnotesize\ttfamily,
showstringspaces=false,
showspaces=false,
numbers=left,
numberstyle=\footnotesize,
numbersep=9pt,
tabsize=2,
breaklines=true,
showtabs=false,
captionpos=b,
xleftmargin=0.5cm
}

\usepackage[sort&compress]{natbib}

% hyperref redefines a number of macros, so it should be last.  Empirically,
% doing so eliminates compiler warnings.
\usepackage[bookmarks, colorlinks]{hyperref}

\newcommand{\projn}{\textsc{TurboLog}\xspace}
\newcommand{\ourtitle}{Langauge, Runtime, and Compiler Aware Logging} 

\newcommand{\todo}[1]{{\color{red}#1}}

\newcommand{\eg}{\hbox{\emph{e.g.}}\xspace}
\newcommand{\ie}{\hbox{\emph{i.e.}}\xspace}
\newcommand{\etc}{\hbox{\emph{etc.}}\xspace}

\newcommand\bench[1]{\textsf{\small #1}}
\newcommand{\niceunitkloc}{\,{\small kloc}\xspace}
\newcommand{\niceunitkb}{\,{\small KB}\xspace}
\newcommand{\niceunitmb}{\,{\small MB}\xspace}
\newcommand{\niceunitsec}{\,s\xspace}
\newcommand{\niceunitpct}{\,\%\xspace}

\newcommand{\codelines}[1]{#1\,kloc\xspace} 

\newcommand{\console}[1]{\texttt{\small #1}}

%% Document-specific hyperref options
\hypersetup{
pdftitle={\ourtitle},
    plainpages=false,
    linkcolor=blue, % Overriding these colors to black is somewhat unfortunate 
    %citecolor=black, % b/c the defaults are useful in color.
    citecolor=blue, % b/c the defaults are useful in color.
    filecolor=black,
    urlcolor=black,
    pdfpagelabels
}
\def\sectionautorefname{Section}
\def\subsectionautorefname{Section}

\begin{document}

\toappear{Draft Document -- Do NOT Redistribute.}

% Copyright
%\setcopyright{acmcopyright}
%\setcopyright{acmlicensed}
%\setcopyright{rightsretained}
%\setcopyright{usgov}
%\setcopyright{usgovmixed}
%\setcopyright{cagov}
%\setcopyright{cagovmixed}

% DOI
\doi{yyy}

% ISBN
\isbn{xxx}

%Conference
\conferenceinfo{TBD}{'17 TBD}
\acmPrice{\$15.00}

\title{\ourtitle}

\numberofauthors{1}
\author{
% 1st. author
Mark Marron\\
       \affaddr{Microsoft Research, USA}\\
       \email{marron@microsoft.com}
}

\maketitle

\begin{abstract} 
Logging is a fundamental part of the software development, deployment, and
monitoring lifecycle but logging support is often provided as an afterthought or
via a library API in a languages core runtime. We argue that given the critical
nature of logging in modern development, the unique needs of the APIs involved,
and the opportunities for optimizing it using semantic knowledge, logging should
be included as a central part of the language and runtime designs. This paper
presents a \emph{language level logging} design which includes support for
logging functionality at all levels of a programming language including, syntax,
runtime support, and hooks for DevOps integration.

Using this integrated approach we build a logging system that supports near
zero-costs for disabled log statements, low cost lazy-copying for enabled log
statements, selective persistence of logging output, unified control of logging
output across different libraries, and DevOps integration for use with modern
cloud-based deployments. To evaluate these concepts we provide two
implementations -- one fully integrated into the design of the \emph{fluent}
programming language and a second, which has slightly reduced features and
performance, but is available as a library for Node.js hosted JavaScript
applications.
\end{abstract}

\category{CR-number}{subcategory}{third-level}

% general terms are not compulsory anymore,
% you may leave them out
\terms
term1, term2

\keywords
keyword1, keyword2

\section{Introduction} 
\label{sec:intro}
Logging has always been a important tool for software developers in
understanding their applications. However, as DevOps oriented workflows have
become more prevalent, logging is becoming an even larger consideration when
building applications. A key area driving this shift is the use of cloud-based
applications and the integration of application monitoring dashboards, such as
Stack Driver~\cite{StackDriver}, NSolid~\cite{NSolid}, or
AppInsights~\cite{AppInsights}, which ingest logs from an application, correlate
this information with other aspects of the system, and provide this in a useful
dashboard format for developers. The additional value provided by these
dashboards and the ability to quickly act on this data makes the inclusion of
rich logging data an integral part of an applications development.

Existing logging library implementations, as provided via core or third party
libraries, are unable to satisfactorily meet the demands of logging in modern
applications. As a result developers must use existing libraries with care to
limit undesirable logging related performance impacts, work to direct logging
output from core or third-party modules to the appropriate channels, and figure
out how to effectively parse the data that is written from various sources.
Consider the following sample JavaScript code which illustrates a number of
concrete issues encountered by Node.js~\cite{Node} developers today.

\begin{figure*}[t]
\lstinputlisting[language=JavaScript,basicstyle=\small]{introExample.js}
\label{sec:introExample}
\caption{Example logging usage in JavaScript.}
\end{figure*}

The first is the challenge of different loggers being used in various parts of
the code in this case \texttt{console.log} writing to the \texttt{stdout} and a
popular framework called \texttt{Winston} which has been configured to write to
a file. As a result some log output will appear on the console while other
output will end up in a file. Further, if a developer changes the logging output
level for \texttt{Winston}, from say \texttt{info} to \texttt{warn}, this will
not change the output level of the \texttt{console} output. Developers can work
around this to some degree by enforcing the use of a single logging framework
for their code but they will not always be able to control the frameworks used
by external libraries.

The next challenge comes from the ad-hoc nature of log message formatting. In
most cases the logging API's provided for developers are setup to take a format
string and some formattable arguments. This is great for quick and easy outputs
but quickly leads to a plethora of ad-hoc formats that cannot be easily parsed
and loaded into dashboard or analytics tooling. Modern logging frameworks,
log4j~\cite{log4j}, Winston~\cite{Winston}, Bunyan~\cite{Bunyan}, etc. provide
some support for consistently formatting ans structuring output but
fundamentally this problem is left as a problem development teams need to solve
via coding conventions and reviews.

In addition to these functionality issues there are also performance problems
that plague existing logging solutions. One concern is that even if a given
logging level is disabled, as \texttt{debug} and \texttt{trace} levels usually
are, the code to generate and format the log message is still executed. This can
lead to code that looks like it will not be executed but that in reality incurs
large parasitic costs. This can be seen in the \texttt{logger.debug} statement
in the example, which at the default level does not print to the log, but will
still result in the creation of the literal object and generation of a format
string on every execution of the loop. This cost leads developers to defensively
remove these statements from code instead of depending on the runtime to
eliminate their costs when deploying the application.

Finally, there is the odd issue that, in many cases a developer only cares about
much of the data in the log if/when there is actually an issue that needs to be
investigated. In other cases this information represents pure execution overhead
for the application. An example of this is the \texttt{logger.info} message
about the args and result of the \texttt{check} call. In the case of a
successful execution the content of this log statement is not useful and the
cost of producing this plus the increased log footprint is pure overhead.
However, if the \texttt{check} statement fails then having this information
about what events led up to the failure may be critical in diagnosing/fixing the
issue. In current logging frameworks this is an unavoidable conundrum and, in
any case where the trace history is needed, the logging statements must be
added.

To address these issues we propose a new approach, \emph{language level
logging}, in which logging is viewed as a first class feature in the
design/implementation of a programming language and runtime instead of simply
another library to be included. Taking this view enables us to leverage language
semantics, focused compiler optimizations, and semantic knowledge in the runtime
to provide a uniform and high performance logging API.

\noindent
The contributions of this paper include:
\begin{itemize}
\item The view that logging is a fundamental aspect of programming and should be
included as a first class part of language, compiler, and runtime design.

\item A novel logging technique that uses immutability semantics in the
programming language to enable ultra-low cost logging which is
5$\times$-100$\times$ faster than existing approaches.

\item A novel dual-level approach to log generation and writing that allows a
programmer to log execution data eagerly but only pay the cost of writting it to
the log if it turns out to be interesting/relevant.

\item Implementation of this technique in a new programming language,
\emph{fluentpl}, which fully realizes the logging ideas in this paper. 

\item An implementation in Node.js with the ChakraCore~\cite{NodeChakraCore} 
JavaScript engine to demonstrate that key ideas can be applied to existing
languages/runtimes and provide an production implementation for use in
performance evaluations.
\end{itemize}

\section{Design}
\label{sec:design}
% !TeX root = LanguageLevelLogging.tex
This section describes opportunities, using language, runtime, or compiler support, to address 
general challenges surrounding logging outlined in \autoref{sec:intro}. We can roughly divide 
these into two classes -- performance oriented and functionality oriented. 

\subsection{Logging Performance}
\label{subsec:performancedesign}

\begin{design}
The cost of a disabled logging statement, one that is at a logging level that is disabled, 
should have zero-cost at runtime. This includes both the direct cost of the logging action 
and the indirect cost of of building a format string and processing any arguments. 
\end{design}

When logging frameworks are included as libraries the compiler/JIT does not, 
in general, have any deep understanding of the enabled/disabled semantics of 
the logger. As a result the compiler/JIT will not be able to fully eliminate 
dead-code associated with disabled logging statements and will pay, individually 
small but widespread, parasitic costs for these disabled logging statements. 
These costs can be very difficult to diagnose, as they are widely dispersed and 
individually small, but can add up to several percentage points of application 
runtime.To avoid these parasitic costs we propose including logging primitives 
in the core specification of the programming language or, if that is not possible, 
adding compiler/JIT specializations to support them. 

An additional advantage of lifting log semantics to the language specification 
level is the ability to statically verify logging uses. Common errors include 
format specifier violations~\cite{tyepcheckprintf} and accidental state 
modification in the logging message computation. If the language semantics 
specify logging API's then both of these error classes can be statically 
checked to avoid runtime errors or hisenbugs that appear/disappear when logging 
levels are changed.

\begin{design}
The cost of an enabled logging statement has two components -- (1) the cost to 
compute the set of arguments to the log statement and (2) the cost to format and 
write this data into the log. The cost of computing the argument values is, in 
general unavoidable, and must be done on the hot path of execution. However, 
the cost of (2) should be reduced and/or moved off hot path as much as possible.
\end{design}

To minimize the cost of computing arguments to the log statement and speed their 
processing we propose a novel log format specification mechanism using 
\emph{preprocessed} and stored log formats along with a set of \emph{log expandos} 
which can be used a shorthand in a log to specify common, but expensive/complicated, 
value to compute. The use of preprocessed format messages allows us to save time, 
the type checking and processing of each argument does not require parsing the 
format string, and instead of eagerly stringifying each parameter we can do a quick 
immutable copy to be formatted later if needed. Expandos provide convinient 
ways to add data into the log, such as the current date/time, the host IP, or a 
current request ID, that would either be more expensive or more awkward to compute 
explicitly on a regular basis.

\subsection{Logging Functionality}
\label{subsec:functionalitydesign}
\begin{design}
Logging serves two related, but slightly conflicting roles, in modern systems 
a logger should support both of them simultaniously without comprimising the 
effectiveness of either one. The first role is to provide detailed information 
on the sequence of events preceeding a bug to aid the developer in tiraging and 
reproducing the issue. The second role is to provide general telemetry 
information and visibility into the overall behavior of the application. 
\end{design}

To support these dual roles we propose a dual-level logging approach. In the 
first level all messages are initially stored, as a format + immutable 
arguments, into an in-memory buffer. This operation is high performance and 
suitable for high frequency writes of detailed logging information needed for 
debugging. Further, in event an error is encountered the full contents of 
detailed logging can be flushed to aid in debugging. In the second level these 
detailed messages can be filtered out and only the high-level telemetry focused 
messages can be saved, formatted, and written into the stable log. This filtering 
avoids the pollution the saved logs with overly detailed information while 
preserving the needed data for monitoring the overall status of the application. 

\begin{design}
Logging code should not obscure the logic of the application that it is 
supporting. Thus, a logger should provide specialized logging primitives 
that cover common cases, such as conditional logging, that would otherwise 
require a developer to add new logic flow into their application specifically 
for the logger.
\end{design}

Common scenarios that often involve additional control or data flow logic 
include \emph{conditional logging} where a message is only written when a 
specific condition is satisfied, \emph{child loggers} which handle a specific 
subtask and often developers want to include additional information in all 
log messages from this subtask, and \emph{bracketing entries} where 
developers want to mark the start/end of something and include correlated 
timing (and other) information in the bracketing. All of these scenarios 
involve the developer adding additional, error-prone, control and data flow 
to the program which obscures the core algorithmic code. Thus, we propose 
adding primitive methods for supporting all of these scenarios without requiring 
additional developer implemented logic.

\noindent
Challenges integrating log data from different sources and difficulty in post processing.
\begin{enumerate}
\item Difficulty in specifying uniform and appropriate logging levels across 
    multiple packages -- and quite possibly multiple logging frameworks.
\item Difficulty in ensuring all logging data is written to a consistent location 
    across multiple packages -- and quite possibly multiple logging frameworks.
\end{enumerate}



\section{Implementation}

\section{Evaluation}

\section{Conclusion}

\section{Notes} log output scheduling -- critical for devops/cloud integration
-- should be background task


\balance

{
\raggedright 

\bibliographystyle{abbrv}
\bibliography{bibfile} 
}


\end{document}
